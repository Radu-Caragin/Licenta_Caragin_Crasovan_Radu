\thispagestyle{abstractpagestyle}

\vspace*{36pt}

\begin{center}
\textbf{\fontsize{20pt}{24pt} \selectfont REZUMAT}
\end{center}

\vspace{24pt}

Sistemul de automatizare al casei bazat pe placa de dezvoltare Arduino Mega dispune de mai multe ansamble inteligente menite să gestioneze eficiența locuinței. Sistemul de stabilizare a temperaturii monitorizează și reglează temperatura din casa. Acesta este alcătuit dintr-un senzor de temperatură, un ventilator pentru răcire și un bec pentru încălzire. Utilizatorii setează temperatura ambientală dorită cu ajutorul unei tastaturi astfel, le este oferit un mediu confortabil.

Pentru asigurarea siguranței, proiectul include un sistem de detectare al gazelor și al fumului dotat cu alarmă. Acest sistem utilizează un buzzer pentru a semnala apariția iregularităților și un senzor de detectare a gazelor și a fumului. Menirea acestui sistem este protejarea împotriva incendiilor și prevenirea accidentelor provocate de scurgerile de gaze.

Iluminatul este bazat pe un sistem autonom format din led-uri ce se aprind la detectarea mișcării și în funcție de nivelul de lumină ambientală. Modulul cu fotorezistor detectează nivelul de lumină din cameră și ajustează iluminatul, economisind energie. De asemenea, un alt led se aprinde automat la detectarea mișcării, îmbunătățind accesibilitatea și siguranța locuinței.

Un alt ansamblu al proiectului este reprezentat de un sistem de deschidere automată a ușii, format dintr-un servomotor și un senzor ultrasonic. Servomotorul este comandat de senzorul ultrasonic, acesta trimițând un semnal la detectarea persoanelor din vecinătatea ușii. Acest sistem facilitează accesul în casă și sporește nivelul de confort.

Datele de la senzorul de temperatură și umiditate, senzorul de gaze și fum și senzorul de presiune barometrică și altitudine sunt afișate pe un display LCD, oferind o vizualizare locală și clară a parametrilor monitorizați.

Proiectul facilitează monitorizarea parametrilor la diferite distanțe. Pentru distanțe medii, este folosit un modul Bluetooth ce trimite datele către o aplicație Android implementată special pentru acest sistem. Pentru distanțe mari, se utilizează un modul NodeMCU cu ESP8266 ce facilitează conexiunea prin WiFi, astfel parametrii pot fi vizualizați utilizând platforma Blynk, atât în varianta desktop cât și mobile.

Toate aceste sisteme sunt integrate într-o machetă construită de mine, demonstrând funcționalitatea și eficiența proiectului de automatizare a casei.


%Rezumatul este destinat să informeze despre conținutul lucrării printr-o scurtă descriere a cercetării de maximum o pagină, a procedurilor/metodelor, precum și a rezultatelor sau concluziilor acesteia. Rezumatul în limba română devine obligatoriu pentru lucrările editate în alte limbi decât limba română și se va scrie cu caractere Arial de 12 pt. Acesta va începe la două rânduri lăsate libere după titlul "REZUMAT". Înainte de titlu se vor lăsa libere trei linii de 12 pt.

\vfill