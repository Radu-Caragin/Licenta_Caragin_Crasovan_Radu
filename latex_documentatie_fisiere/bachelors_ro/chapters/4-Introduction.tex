\chapter{Introducere}\label{section:introduction}
\thispagestyle{pagestyle}

\section{Domeniul abordat}
Proiectul realizat de mine se concentrează pe controlul unui sistem încorporat (embedded system) utilizat în tehnologia de automatizare a casei ce comunică cu platforme web și mobile prin intermediul conexiunilor WiFi și Bluetooth. Acest domeniu este foarte des folosit pentru a oferi utilizatorilor confort, dar și securitate, asigurând o modalitate de a supraveghea în timp real condițiile esențiale și diverși parametrii ai locuinței. Tehnologiile cele mai folosite în acest domeniu sunt: IoT (Internet of Things), HMI (Human-Machine Interface) și SCADA (Supervisory Control and Data Acquisition) \cite{riffat}.

\section{Contextul lucrării}
Pe măsură ce tehnologia a evoluat aceasta a avut un impact semnificativ asupra vieții omului, iar sistemele de automatizare a casei reprezintă un exemplu ideal pentru acest context. Automatizarea casei presupune integrarea unor tehnologii de control și gestionare automată a funcțiilor unei case, de la iluminat și mecanisme automate, până la monitorizare și securitate. Datorită acestor sisteme, viața oamenilor devine considerabil mai ușoară, iar factorul principal oferit îl reprezintă confortul. Totodată, intervine și factorul economisirii energiei, deși sistemul propriu-zis este un consumator, beneficele oferite spre exemplu cele de iluminat, compensează acest fapt.

Un alt factor semnificativ oferit de aceste sisteme este siguranța. Cu ajutorul anumitor senzori ce sunt capabili să monitorizeze constat diverși parametrii ai mediului înconjurător, se pot detecta diferite neregularități ce pot apărea în perimetrul lcouinței. O dimensiune a acestor sisteme ce nu trebuie omisă este deschiderea de noi orizonturi pentru persoane cu dizabilități sau vârstnici ce doresc să trăiască independent.

Motivul pentru care am ales acest proiect este reprezentat de faptul că oferă o oportunitate în dobândirea și aplicarea cunoștințelor în mai multe domenii ale tehnologiei. Proiectele de sisteme încorporate sunt un mediu propice pentru a explora și a împreuna domenii precum hardware, software și electronică, trei domenii care mă pasionează. Un alt motiv personal este reprezentat de faptul că în urma realizării acestui proiect aș vrea ca acesta să fie implementat în rulota tatălui meu.

Pentru a realiza lucrarea, am folosit ediotrul de documente \texttt{LaTeX}. Acesta este folosit în general pentru lucrări științifice și academice, acesta oferă precezie în redectarea lucrărilor, dar și control complet asupra formatării.